\documentclass[11pt]{article}
\usepackage{amssymb}

\textwidth = 6.5 in
\textheight = 9 in
\oddsidemargin = 0.0 in
\evensidemargin = 0.0 in
\topmargin = 0.0 in
\headheight = 0.0 in
\headsep = 0.0 in
\parskip = 0.2in
\parindent = 0.0in


\title{Thesis Plan}
\author{Dave Peixotto}
\begin{document}
\maketitle


%%
% INTRODUCTION
%%
\section{Implementation Work}
\begin{enumerate}
\item get rematerialization working (in progress)

\item implement smart splitting - after splitting a live range you
should prune the basic blocks that are included in the live range but
are on a path along which there is never a use.

\item Heuristics for splitting to improve allocation time.\\
  \begin{description}
  \item[chow] (used after a color has been assigned to a live range and we
  are checking neighbors to see if they need to be split) 
  split a live range when the number of neighbors is greater than
  twice the number of remaining colors.
  \item[chow] (used during splitting) do not add a basic block if it brings
  in too many new neighbors. Chow uses the metric that the number of
  neighbors should be less than or equal to the number of colors left
  before the addition of the block.

  \item[ours] (used during splitting) Do not add a block if it makes the
  number of forbidden colors to great (similar to heuristic (b)
  above).
  \end{description}



\item Don't flush temporary register contents when the basic block has
only one successor and the successor has only one predecessor.

\item select better color when assigning color
  \begin{enumerate}
  \item select color already in forbidden set of a neighbor
  \item select color that has been assigned to a live range from which
  we were split.
  \end{enumerate}

\item when loading a use that has been spilled for an instruction
that is a copy then just load directly into the source instead of a
temporary and then copy.

\item live range coalescing

\item Alternate splitting algorithm. \\
The idea is to extend Chow's splitting algorithm to allow for
different choices to be made. Chow does not allow for choices to be
made along the fringe of the BFS. One way to extend the splitting is
to allow for different choices to be made for the blocks to include in
the new live range
  \begin{enumerate}
  \item search both up the graph and down the graph for other blocks
  in the live range that can be added.
  \item search down the graph to include as many uses as possible with
  a def, do not include other defs unless they are ``harmless''
  \item Use some kind of cost function to decide whether or not to
  include a block. You could try including blocks that contain the
  most uses (high spill cost), or try including blocks that pull in
  the least amount of forbidden registers.
  \end{enumerate}
\end{enumerate}

\section{Experiments}
\begin{enumerate}
\item Gather numbers on how often splitting helps you. We would like to
know when you split a live range whether you end up being able to
color part of it when you would have had to spill the entire thing if
you had not split.

\item Gather statistics for how chow performs by varying different
paramerters:
\begin{enumerate}
\item basic block length
\item number of registers
\item enhanced code motion
\item rematerialization
\item live range coalescing
\end{enumerate}

\item Run adaptive Chow and see if it can come up with a better
allocation.

\end{enumerate}

\section{Adaptive Chow}
Parameters
\begin{enumerate}
\item basic block length
\item priority function: adjusting weights: \texttt{LOOP-DEPTH,
LOAD-SAVE, STORE-SAVE, MOVE-SAVE}.
\item split small priority live ranges (including possibly negative
priority live ranges)
\item number of temporary registers reserved
\item decrease priority on clustered live ranges 
\item use ``Enhanced Code Motion''
\item only use ``Enhanced Code Motion'' on blocks with exactly one
predecessor to catch exactly the case of a load/store pair.
\item use rematerialization
\item use enhanced splitting (prune useless live units)
\item use alternate splitting (search agressively for uses)
\item heuristics for improving allocation time ($\times 3$)
\end{enumerate}


\section{Our Practical Improvements}
\begin{enumerate}
\item (done) Enhanced Code Motion - replace a load/store pair with a register
to register copy.
\item (done) building inital live ranges with SSA
\item better splitting
\item rematerialization
\item coalescing
\end{enumerate}

\section{Questions}
\begin{enumerate}
\item what about caller/callee save? can we say that we are doing
machine independent register allocation?

\item does ra say a double takes two float registes? (i hope not)
\end{enumerate}

\section{Writing}
\subsection{Thesis}
\begin{enumerate}
\item Classic Chow
\item Enhanced Chow
\item Adaptive Chow
\item Implementation Details
\item Comparison with Chaitin
\end{enumerate}

\subsection{Papers}
Possible places to submit results
\begin{enumerate}
\item SP\&E
\item
\item
\end{enumerate}

\section{Timeline}
\begin{description}
\item[3 weeks] (1 June)  Implementation items 1-7
\item[1 weeks] (8 June)  Implemention item 8
\item[2 weeks] (22 June) Adaptive compiler setup (not sure what to do)
\item[1 weeks] (29 June) Experiments complete
\item[5 weeks] (3 August) Thesis writing
\end{description}
\end{document} 
